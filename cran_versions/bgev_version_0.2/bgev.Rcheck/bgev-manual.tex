\nonstopmode{}
\documentclass[a4paper]{book}
\usepackage[times,inconsolata,hyper]{Rd}
\usepackage{makeidx}
\makeatletter\@ifl@t@r\fmtversion{2018/04/01}{}{\usepackage[utf8]{inputenc}}\makeatother
% \usepackage{graphicx} % @USE GRAPHICX@
\makeindex{}
\begin{document}
\chapter*{}
\begin{center}
{\textbf{\huge Package `bgev'}}
\par\bigskip{\large \today}
\end{center}
\ifthenelse{\boolean{Rd@use@hyper}}{\hypersetup{pdftitle = {bgev: Bimodal GEV Distribution with Location Parameter}}}{}
\ifthenelse{\boolean{Rd@use@hyper}}{\hypersetup{pdfauthor = {Cira Otiniano Author; Yasmin Lirio Author}}}{}
\begin{description}
\raggedright{}
\item[Type]\AsIs{Package}
\item[Title]\AsIs{Bimodal GEV Distribution with Location Parameter}
\item[Version]\AsIs{0.2}
\item[Date]\AsIs{2025-11-04}
\item[Depends]\AsIs{EnvStats, DEoptim, stats, MASS}
\item[Description]\AsIs{Density, distribution function, quantile function random generation and estimation of bimodal GEV distribution given in Otiniano et al. (2023) <}\Rhref{https://doi.org/10.1007/s10651-023-00566-7}{doi:10.1007/s10651-023-00566-7}\AsIs{>. This new generalization of the well-known GEV (Generalized Extreme Value) distribution is useful for modeling heterogeneous bimodal data from different areas.}
\item[License]\AsIs{GPL-3}
\item[NeedsCompilation]\AsIs{no}
\item[Author]\AsIs{Cira Otiniano Author [aut],
Yasmin Lirio Author [aut],
Thiago Sousa Developer [cre]}
\item[Maintainer]\AsIs{Thiago Sousa Developer }\email{thiagoestatistico@gmail.com}\AsIs{}
\end{description}
\Rdcontents{Contents}
\HeaderA{bgev-package}{Bimodal GEV Distribution with Location Parameter}{bgev.Rdash.package}
\aliasA{bgev}{bgev-package}{bgev}
\aliasA{dbgev}{bgev-package}{dbgev}
\aliasA{pbgev}{bgev-package}{pbgev}
\aliasA{qbgev}{bgev-package}{qbgev}
\aliasA{rbgev}{bgev-package}{rbgev}
\keyword{bgev}{bgev-package}
\keyword{gev}{bgev-package}
\keyword{bimodal}{bgev-package}
\keyword{distribution}{bgev-package}
%
\begin{Description}
Density, distribution function, quantile function random generation and estimation of bimodal GEV distribution given in Otiniano et al. (2023) <doi:10.1007/s10651-023-00566-7>. This new generalization of the well-known GEV (Generalized Extreme Value) distribution is useful for modeling heterogeneous bimodal data from different areas.
\end{Description}
%
\begin{Usage}
\begin{verbatim}
dbgev(y, mu = 1, sigma = 1, xi = 0.3, delta = 2)
pbgev(y, mu = 1, sigma = 1, xi = 0.3, delta = 2)
qbgev(p, mu = 1, sigma = 1, xi = 0.3, delta = 2)
rbgev(n, mu = 1, sigma = 1, xi = 0.3, delta = 2)
\end{verbatim}
\end{Usage}
%
\begin{Arguments}
\begin{ldescription}
\item[\code{y}] 
a unidimensional vector containing the points to compute the density (dbgev) or  
the probability (pbgev) froma bimodal GEV distribution with parameters
\code{mu}, \code{sigma}, \code{xi} and \code{delta}.

\item[\code{p}] 
a unidimensional vector containing the probabilities 
used to compute the quantiles

\item[\code{n}] 
an integer describing the number of observations to generate random bimodal
GEV observations

\item[\code{mu}] 
location parameter

\item[\code{sigma}] 
shape parameter

\item[\code{xi}] 
shape parameter

\item[\code{delta}] 
bimodality parameter

\end{ldescription}
\end{Arguments}
%
\begin{Details}
Density, distribution function, quantile function and random generation of 
bimodal GEV distribution with location parameter. In addition, maximum likelihood
estimation based on real data is also provided.
\end{Details}
%
\begin{Value}
\code{dbgev} gives the density, \code{pbgev} gives the distribution function, 
\code{qbgev} gives the quantile function, and \code{rbgev} generates random
bimodal GEV observations. 
\end{Value}
%
\begin{Note}

The probability density of a GEV random variable; \eqn{Y \sim F_{\xi, \sigma, \mu}}{} is given by: 

\deqn{
	f_{\xi, \mu, \sigma}(y)=
		\begin{cases}
		\dfrac{1}{\sigma} \left[ 1+ \xi \left(\dfrac{y-\mu}{\sigma}\right) \right]^{(-1/\xi) -1} \exp\left\{- \left[1+\xi\left(\dfrac{y-\mu}{\sigma}\right)\right]^{-1/\xi}\right\} ,& \text{if } \xi \ne 0 \\
		\dfrac{1}{\sigma} \exp \left\{ - \left(  \dfrac{y-\mu}{\sigma}\right)  - \exp \left[ - \left(  \dfrac{y-\mu}{\sigma}\right)  \right]  \right\},              & \text{if }  \xi = 0 ,
	\end{cases}
}{}
where \eqn{\xi}{} and \eqn{\sigma}{} are the shape parameters and \eqn{\mu}{} is the location parameter.

The bimodal Generalized Extreme Value (GEV) model, denoted as BGEV, consists of composing the distribution of a random variable following the GEV distribution with a location parameter \eqn{\mu=0}{}, i.e., \eqn{Y \sim F_{\xi, 0, \sigma}}{}, with the transformation \eqn{T_{\mu, \delta}}{} defined below. Thus, the cumulative distribution function (CDF) of a random variable BGEV, denoted as \eqn{X \sim F_{BG_{\xi, \mu, \sigma, \delta}}}{}, is given by:

\deqn{F_{BG_{\xi,\mu,\sigma, \delta}}(x) =  F_{\xi, 0, \sigma}(T_{\mu, \delta}(x)),}{}

where the function \eqn{T_{\mu, \delta}}{} is defined as:

\deqn{T_{\mu, \delta}(x)=\left(  x - \mu  \right) \left| x -\mu \right| ^{\delta}, \quad \delta > -1, \quad \mu \in \mathbb{R}.}{}

Moreover, the function \eqn{T}{} is invertible, with the inverse given by:

\deqn{T^{-1}_{\mu, \delta}(x) = \text{sng}(x) |x|^{\dfrac{1}{\left( \delta +1 \right) }} + \mu.}{}

Additionally, it is differentiable, and its derivative has the following form:

\deqn{T'_{\mu, \delta}(x) = (\delta + 1 ) |x - \mu|^{\delta}.}{}


Its probability density function \eqn{X\sim F_{BG_{\xi,\mu,\sigma, \delta}}}{} is given by

\deqn{
	f_{BG_{\xi,\mu,\sigma, \delta}} (x)= \begin{cases}
		\dfrac{1}{\sigma} \left[ 1+ \xi \left(\dfrac{T_{\mu, \delta}(x)}{\sigma}\right) \right]^{(-1/\xi) -1} \exp\left[- \left[1+\xi\left(\dfrac{T_{\mu, \delta}(x)}        {\sigma}\right)\right]^{-1/\xi}\right] T'_{\mu, \delta}(x)
		, & \xi \neq 0 \\
		\dfrac{1}{\sigma} \exp \left( - \dfrac{T_{\mu, \delta}(x)}{\sigma} \right) \exp \left[- \exp \left( - \dfrac{T_{\mu, \delta}(x)}{\sigma}\right)  \right] T'_{\mu, \delta}(x), &  \xi=0.
	\end{cases}
}{}

\end{Note}
%
\begin{Author}
Cira Otiniano Author [aut],
  Yasmin Lirio Author [aut],
  Thiago Sousa Developer [cre]

Maintainer: Thiago Sousa Developer <thiagoestatistico@gmail.com>
\end{Author}
%
\begin{References}
Otiniano, Cira EG, et al. (2023). A bimodal model for extremes data. Environmental and Ecological Statistics, 1-28. http://dx.doi.org/10.1007/s10651-023-00566-7
\end{References}
%
\begin{Examples}
\begin{ExampleCode}
# generate 100 values distributed according to a bimodal GEV
x = rbgev(50, mu = 0.2, sigma = 1, xi = 0.5, delta = 0.2) 
# estimate the bimodal GEV parameters using the generated data
bgev.mle(x)
\end{ExampleCode}
\end{Examples}
\HeaderA{bgev.mle}{Parameter estimation of bimodal GEV distribution based on real data  that appears to be bimodal.}{bgev.mle}
\keyword{bgev}{bgev.mle}
\keyword{gev}{bgev.mle}
\keyword{bimodal}{bgev.mle}
\keyword{mle}{bgev.mle}
\keyword{estimation}{bgev.mle}
%
\begin{Description}
Finds the maximum likelihood estimators of the bimodal GEV distribution 
parameters by minimizing the log-likelihood function of the data.

The estimation procedure is a divided into three steps:

1) Use egevd for starting values: We estimate the data using a unimodal GEV distribution (function egevd from package EnvStats). This is justified 
by the fact that the bgev distribution with delta = 0 is a unimodal GEV distribution with parametesr mu sigma and xi.  

2) DEOptim for more adequate starting values: Use the starting values to create lower and upper bounds and feed it into DEOptim to get starting values for all
parameters simutaneously.

3) Use optim for local search with starting values for DEOptim.

\end{Description}
%
\begin{Usage}
\begin{verbatim}
bgev.mle(x, method_envstats = "mle", deoptim.itermax = 200, 
optim.method = "L-BFGS-B",start = NULL, lower = NULL, upper = NULL) 
\end{verbatim}
\end{Usage}
%
\begin{Arguments}
\begin{ldescription}
\item[\code{x}] 
a unidimensional vector containing observations to estimate a bimodal GEV 
distribution

\item[\code{start}] 
a vector of dimension 4 containing the starting values for each of the bimodal GEV
parameters where the search is going to take place. 


\item[\code{lower}] 
a vector of dimension 4 containing the lower limit for each of the bimodal GEV
parameters where the search is going to take place. 

\item[\code{upper}] 
a vector of dimension 4 containing the upper limit for each of the bimodal GEV
parameters where the search is going to take place. 

\item[\code{method\_envstats}] 
method to use in the egevd function from envstats package.

\item[\code{deoptim.itermax}] 
number of iterations in the deoptim step.

\item[\code{optim.method}] 
maximum number of interations when finding a good starting value for the optimization algorithm.

\end{ldescription}
\end{Arguments}
%
\begin{Value}
A list with components returned by the optim R function for which the 
main values are
\begin{ldescription}
\item[\code{par}] best parameters of bimodal gev fitting the data
\item[\code{value}] value of the log-likelihood corresponding to 'par'
\end{ldescription}
\end{Value}
%
\begin{Author}
Cira Otiniano Author [aut],
  Yasmin Lirio Author [aut],
  Thiago Sousa Developer [cre]
\end{Author}
%
\begin{References}
OTINIANO, Cira EG, et al. (2023). A bimodal model for extremes data. Environmental and Ecological Statistics, 1-28. http://dx.doi.org/10.1007/s10651-023-00566-7
\end{References}
%
\begin{Examples}
\begin{ExampleCode}
# generate 100 values distributed according to a bimodal GEV
x = rbgev(50, mu = 0.2, sigma = 1, xi = 0.5, delta = 0.2) 
# estimate the bimodal GEV parameters using the generated data
bgev.mle(x)
\end{ExampleCode}
\end{Examples}
\HeaderA{bgev.support}{Support of the bimodal GEV distribution}{bgev.support}
\keyword{bgev}{bgev.support}
\keyword{gev}{bgev.support}
\keyword{bimodal}{bgev.support}
\keyword{support}{bgev.support}
%
\begin{Description}
When the shape parameter \code{xi} is different from zero, the support
is truncated either at the left or at the right side of the real. 
Considering the support is particularly useful to estimating momoments and 
to compute the likelihood function.
\end{Description}
%
\begin{Usage}
\begin{verbatim}
bgev.support(mu, sigma, xi, delta)
\end{verbatim}
\end{Usage}
%
\begin{Arguments}
\begin{ldescription}
\item[\code{mu}] 
location parameter

\item[\code{sigma}] 
shape parameter

\item[\code{xi}] 
shape parameter

\item[\code{delta}] 
bimodality parameter

\end{ldescription}
\end{Arguments}
%
\begin{Value}
It returns a vector representing the interval for which the density function 
is not null. 
\end{Value}
%
\begin{Note}
The Support of the bimodal GEV distribution is given by the following equation:

\deqn{
    \begin{cases}
     x \in \mathbb {R} : x \geq \mathbf{sng} \left(-\frac{\sigma}{\xi}\right) | \frac{\sigma}{\xi}| ^{\frac{1}{\delta+1}}+\mu, &  \xi \neq 0 \\
     x \in \mathbb {R}, & \xi =0.
    \end{cases}
}{}


\end{Note}
%
\begin{Author}
Cira Otiniano Author [aut],
  Yasmin Lirio Author [aut],
  Thiago Sousa Developer [cre]
\end{Author}
%
\begin{References}
Otiniano, Cira EG, et al. (2023). A bimodal model for extremes data. Environmental and Ecological Statistics, 1-28. http://dx.doi.org/10.1007/s10651-023-00566-7
\end{References}
%
\begin{Examples}
\begin{ExampleCode}
# Computes the support of a specific bimodal GEV distribution
support = bgev.support(mu=1, sigma=10, xi=0.3, delta=2)
\end{ExampleCode}
\end{Examples}
\HeaderA{distCheck}{Check implementation of a distribution in \R{}{}. }{distCheck}
\keyword{distribution}{distCheck}
\keyword{test}{distCheck}
\keyword{implementation}{distCheck}
%
\begin{Description}
This is just a refactorization of 
the original \R{} function \code{distCheck} to allow customization for heavy tailed distributions
and distribution with constrained support. Assumes one has implemented density, 
probability, quantile and random generation for a new distribution. 
Originally implemented to be used with the bimodal GEV distribution
\end{Description}
%
\begin{Usage}
\begin{verbatim}
distCheck(fun = "norm", n = 1000, robust = FALSE, subdivisions = 1500, 
support.lower = -Inf, support.upper = Inf, 
var.exists = TRUE, print.result = TRUE, ...)
\end{verbatim}
\end{Usage}
%
\begin{Arguments}
\begin{ldescription}
\item[\code{fun}] 
Name of distribution. E.g, "bgev" for bimodal GEV, "gev" for GEV, "norm" for normal. 

\item[\code{n}] 
size of the sample when generating random values through "rfun"

\item[\code{robust}] 
Boolean. Used for computing mean and variance in a robust way when evaluating mean 
and variance.

\item[\code{subdivisions}] 
used for numerical integration when using "dfun"

\item[\code{support.lower}] 
lower limit of the support of the distribution

\item[\code{support.upper}] 
upper limit of the support of the distribution

\item[\code{var.exists}] 
Boolean indicating if variance exists (useful for gev, bimodal gev, stable etc)

\item[\code{print.result}] 
Boolean indicating weather to print results of the tests on screen.

\item[\code{...}] 
parameters passed to the distribution, e.g., mu and sd for normal, mu, sigma, xi
and delta for bimodal GEV.

\end{ldescription}
\end{Arguments}
%
\begin{Details}
All tests results are organized into a list to make it easier to find where 
results went wrong and to make it scalable for testing distributions for 
several parameters. 
\end{Details}
%
\begin{Value}
A list of lists with test results


\code{list( functionTested = fun,
              functionParam = list(...),}\\{}
\code{test1.density =      list(computed = NULL, expected = NULL, error.check = NULL),}\\{}
\code{test2.quantile.cdf = list(computed = NULL, expected = NULL, error.check = NULL),}\\{}
\code{test3.mean.var =     list(computed = list(mean = NULL, var = NULL, log = NULL), }\\{}
\code{ expected = list(mean = NULL, var = NULL, log = NULL),}\\{}
\code{  error.check = NULL, }\\{}
\code{  condition.is.var.finite = TRUE))}
\begin{ldescription}
\item[\code{functionParam}] Additional parameters passed to the distribution with the ... argument
\item[\code{test1.density}] results for the density function test. A list with the 'computed' value 
using the density function and numerical integration, the 'expected' value (which is 1) and 
the result of the 'error.check' comparing both values.
\item[\code{test2.quantile.cdf}] results for the quantile function test. A list with the 'computed' value 
using the quantile function at chosen probabilites on [0,1], the 'expected' value (which is 1) and 
the result of the 'error.check' comparing both values.

\item[\code{test3.mean.var}] results for the random generation and the density function. 
A list with the 'computed' values using the random quantities (computes mean(X), variance(X) and E(log(abs(X))) where X follows
the distribution being tested. The 'expected' is a list containing the same quantities computed using numerical integration

\item[\code{error.check}] check if expected variance and computed variance are close enough.

\item[\code{condition.is.var.finite}] For some distributions there are known conditions for the variance to be finite. This quantity
is expected to be evaluated outside the DistCheck function and passed through the argument 'var.exists'.

\end{ldescription}
\end{Value}
%
\begin{Author}
Cira Otiniano Author [aut],
  Yasmin Lirio Author [aut],
  Thiago Sousa Developer [cre]
\end{Author}
%
\begin{References}
Otiniano, Cira EG, et al. (2023). A bimodal model for extremes data. Environmental and Ecological Statistics, 1-28. http://dx.doi.org/10.1007/s10651-023-00566-7
\end{References}
%
\begin{Examples}
\begin{ExampleCode}
# generate random values for the parameters and test the 
# bimodal gev distribution implementation
set.seed(1)
mu = runif(1,-2,2)
sigma = runif(1,0.1,2)
xi = runif(1,0.3,0.9) * sign(runif(1,-1,1))
delta = 1#runif(1,-0.6,2)
support = bgev.support(mu, sigma, xi, delta)
var.exists = ( xi != 0) & (xi < (delta + 1)/2)
ret = distCheck(fun="bgev", n = 2000, 
      support.lower = support[1], support.upper = support[2], 
      subdivisions = 5000, mu = mu, sigma = sigma, xi = xi, 
      delta = delta, var.exists = var.exists, print.result = TRUE)
\end{ExampleCode}
\end{Examples}
\HeaderA{likbgev}{Log likelihood function for the bimodal GEV distribution.}{likbgev}
\keyword{bgev}{likbgev}
\keyword{gev}{likbgev}
\keyword{bimodal}{likbgev}
\keyword{likelihood}{likbgev}
%
\begin{Description}
Uses the density function to evaluate the likelihood. This is useful for the
'bgev.mle' function.
\end{Description}
%
\begin{Usage}
\begin{verbatim}
likbgev(y, theta = c(1, 1, 0.3, 2))
\end{verbatim}
\end{Usage}
%
\begin{Arguments}
\begin{ldescription}
\item[\code{y}] 
a unidimensional vector containing the points to compute the log likelihood

\item[\code{theta}] 
bimodal GEV parameters as an ordered pair \code{theta = (mu,sigma,xi,delta)}

\end{ldescription}
\end{Arguments}
%
\begin{Value}
a unidimensional vector containing the computed log likelihood for \code{y}.
\end{Value}
%
\begin{Author}
Cira Otiniano Author [aut],
  Yasmin Lirio Author [aut],
  Thiago Sousa Developer [cre]
\end{Author}
%
\begin{References}
Otiniano, Cira EG, et al. (2023). A bimodal model for extremes data. Environmental and Ecological Statistics, 1-28. http://dx.doi.org/10.1007/s10651-023-00566-7
\end{References}
%
\begin{Examples}
\begin{ExampleCode}
# get random points from bimodal GEV
y = rbgev(100, mu = 1, sigma = 1, xi = 0.3, delta = 2)

# compute log-likelihood
likbgev (y, theta = c(1, 1, 0.3, 2)) 
\end{ExampleCode}
\end{Examples}
\printindex{}
\end{document}
